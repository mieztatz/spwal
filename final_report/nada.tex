\subsection{NaDa}
In the paper \cite{DBLP:conf/conext/ValanciusLMDR09} another Nano Data Center approach called NaDa is proposed. Based on a variant of a peer-to-peer network, NaDa consists of distributed servers managed and controlled by an ISP. According to the authors, devices like DSL or Cable modems can serve as gateways and replace monolithic data centers regarding the use of many Internet services. Their papar, though, only covers video streaming services. 

The authors claim that NaDa consumes up to 30\% less energy than a traditional data center in this use case. The following challenges can be extracted from the paper:

\begin{itemize}
	\item The ISP needs to invest into gateways with more storage and bandwidth to create a reliable network.
	\item There is no research into other applications of NaDa (excep the VoD services).
	\item The users would have to pay for the energy the gateway consumes. This cost is said to be "not significant", but users might not share this opinion.
	\item "[...] each user is assumed to have identical network distance to every other user in a network (this is what would happen on a mid-sized metropolitan area network)." As a consequence, energy consumption and access times could be a lot less promising in rural areas with greater and less evenly spread distances between users. 
	\item The energy savings depend on the number of users, which could make possible users more reluctant while NaDa is not yet widespread.
\end{itemize}
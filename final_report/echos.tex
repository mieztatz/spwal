\subsection{ECHOS}
ECHOS introduces a concept for Nano Data Center that can or should completely replace monolithic data centers \cite{Laoutaris:2008:EEC:1341431.1341442}. The authors call it a radical solution for data management and provision.
According to this concept, so-called "boxes" are set up at the edges of the network, eg. in home gateways (see \cite{technicolor}). These boxes communicate with each other via a peer-to-peer system. The peer-to-peer system as well as the bandwidth is controlled by a central unit, such as the ISP. However, the approach of networking boxes via a peer-to-peer system, and thus providing or sharing content, requires some conditions. So it is first necessary to provide a distributed hosting edge infrastructure. Furthermore, there are still some problems added. In ECHOS these are listed as follows \cite{Laoutaris:2008:EEC:1341431.1341442}:
\begin{itemize}
	\item "Lack of service guarantees due to uncontrolled interface between different application [...]."
	\item "Inefficient use of network's and other peer's resources and consequently supoptiomal performance [...]."
	\item "Even if sufficient status information is in place, still P2P is inherently unable to use it as it was designed around selfish user behavior and free-riding prevention mechanim [...]."
	\item "Absence of security and control make it impossible to guarantee the integrity and security of content."
\end{itemize}
Nowadays, concerns about the environment are increasing and finding alternatives that reduce waste, CO2-emission or energy consumption, is a challenging topic. In order to fight environmental problems, it is essential that sustainable solutions are realized in every possible area. These areas also include internet content provision and storage.\\ Traditionally, monolithic data centers are used to store and manage today's constantly rising mass of data. This centralized approach, though, consumes huge amounts of energy and especially cooling is a severe issue. To overcome these issues and provide a more energy-efficient approach, de-centralized models have been introduced. Among others, the model of nano data centers was advocated. This kind of data center solution is said to be highly energy-efficient while still providing sufficient content availability and uptime.\\ 
Thus, the question arises why no such approach has been realized so far. Also, the media attention towards the topic is quite low, even so the findings presented in the according papers, which will be treated in section \ref{StateOfTheArt}, are very promising. \\
Therefore, the purpose of this paper is to identify challenges that prevent the large-scale realization of nano data centers: 
\begin{itemize}
\item What are political and legal challenges that yet have to be overcome?
\item What are technical challenges that have yet to be overcome, especially:
\begin{itemize} 
\item Are nano data centers really as energy efficient as the previous papers indicate? 
\item If so, are the energy savings as high for all use cases? 
\end{itemize}
\end{itemize}
These questions will be answered in the course of this paper. To achieve this, scientific publications concerning nano data centers, as well as related approaches were researched and compared. Moreover, an interview was conducted with an expert on monolithic data centers, to gain further insight into data center models and real world problems that might not yet have come up or seemed relevant in theory.\\
At first, related papers are presented. After that, the used methodology is described and challenges towards a distributed nano data center infrastructure are discussed. The challenges extracted from relevant papers rise from a variety of categories. In this paper, though, only political, legal and technical challenges are analysed in depth. After presenting our findings, we will evaluate and discuss them. Lastly we provide an outlook on what future work could be done to help with the decision process of whether nano data centers are a realistic future alternative to monolithic data centers.

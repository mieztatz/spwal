Nowadays, concerns about the environment are increasing and finding alternatives that reduce waste, CO2-emission or energy consumption is a challenging topic. In order to fight environmental problems, it is essential that sustainable solutions are realized in every possible area. These areas also include internet content provision and storage.\\ Traditionally, monolithic data centers are used to store and manage today's constantly rising mass of data. This centralized approach, though, consumes huge amounts of energy and especially cooling is a severe issue. To overcome these issues and provide a more energy-efficient approach, de-centralized models have been introduced. Among others, the model of nano data centers was advocated. This kind of data center solution is said to be highly efficient while still providing sufficient content availability and uptime.\\ 
Thus, the question arises why no such approach has been realized so far. Also, the media attention for the topic is quite low, although the findings presented in the according papers, which will be treated in the following section (\autoref{StateOfTheArt}), sound very promising. \\
Therefore, the purpose of this paper is to identify challenges that prevent the large-scale realization of nano data centers. 
\begin{itemize}
\item Are nano data centers as energy efficient as the papers indicate? 
\item Are there high energy savings for all use cases? 
\item What are political and technical challenges that yet need to be overcome? 
\end{itemize}
These questions will be answered in the course of this paper. To achieve this, scientific publications concerning nano data centers as well as related approaches were researched and compared. Moreover, an interview was conducted with a specialist to gain further insight into data center models and real world problems that might not come up or seem relevant in theory.\\
First related papers are presented. After that, the challenges towards a distributed nano data center infrastructure are discussed. This discussion includes the methodology, political and legal challenges, and the technical challenges. The evaluation and discussion follow, and the paper is concluded with future work. 


%Our sources indicate that the following issues are most challenging for nano data centers: ....
% maybe irrelevant for introduction? 
\section{Interview}
Kat: Auf der Webseite des LRZ steht, dass Green IT wichtig ist. Was wurde denn bis jetzt schon erreicht? \\
Das LRZ zählt zu den energieeffizientesten Rechenzentren. Und es gibt dafür zwei Gründe. Der eine Grund, und das ist das was wir der Wissenschaft vermitteln müssen, je mehr Energie wir sparen bei der Erbringung unserer Dienstleistungen, umso mehr können die Wissenschaftler rechnen oder eben andere Dienste in Anspruch nehmen. Das heißt im Endeeffekt nichts anderes, als wenn wir eine Rechnung mit weniger Energie durchführen, dann können wir das Geld, dass wir gespart haben, wieder in andere Dienste investieren. Und damit erklärt sich eigentlich schon, warum sich so etwas wie der Super muc, anstatt dass er wie früher, da hat man einfach gesagt, ein Rechner produziert oder braucht
X Strom für die Berechnung, und man braucht noch einmal so x Strom, vielleicht sogar x*1,5 Strom für die Kühlung oder den Betrieb des Rechners, ja?. Wir sind halt jetzt so, dass wir auch noch immer das x Strom für die Rechnung brauchen, aber nur 10, 15 \% extra für die Kühlung. Weil für die Kühlung verhältnismäßig weniger Energie aufgewendet wird, weil am LRZ sowas wie die normale Luftzugkühlung verwendet wird, die kostet ja nichts. Die hat halt in München, und heute ist ein gutes Beispiel, in der Regel weniger als 35 Grad Celsius. \\
And: Das heißt aber im Umkehrschluss auch, dass der limitierende Faktor eher das Geld ist und nicht die Zeit. Früher hat man ja so schnelle Rechner gebaut, weil man in möglichst wenig Zeit ... \\
Nee, das ist nicht ganz richtig. Die Technologie setzt Grenzen in dem was man pro Zeiteinheit rechnen kann. Aber es ist wahr, dass man von dieser Technologie nur begrenze Stückzahlen kaufen können, weil das Geld nicht ausreicht. Was aber richtig ist, ist der Schluss, wenn Sie es jetzt umgekehrt nehmen, dass wir derzeit in einer Position sind, wo wir Rechner kaufen, die größer sind als der Strom, denn wir dafür überhaupt bezahlen können, d. h. wenn wir jetzt den nächsten SuperMuc anschauen, dann ist es bei dem so, dass wir, wenn wir auf dem diesen Lindpack-Algorithmus rechnen wollen, der bestimmt, wo wir weltweit in dieser Top-500 Liste stehen, dann müssen wir vermutlich andere Systeme anschalten, weil wir für beides nicht genug Strom hätten. Und das sind halt schon interessante Sachen. Es ist aber bei uns auch wieder so, und da kommen wir jetzt zur ersten Frage zurück, ihr wisst, dass so ein Prozessor eine gewisse Taktfrequenz hat, und wir betreiben die Systeme mit [einer] niedrigeren als der normalen Taktfrequenz, weil wir festgestellt haben, dass die meisten Softwarepackete gar nicht die volle Taktfrequenz ausnützen können. \\
And: Dadurch sparen Sie also Strom, obwohl Sie... \\
Genau, richtig. Wenn ich jetzt 10 \% runter gehe mit der Taktfrequenz, dann spare ich mir x \% vom Strom. Und wir sind halt derzeit so, dass wir im Prinzip größenordnungsmäßig irgendwo bei 65 \% von der nominalen Taktfrequenz verwenden. D. h. wir fahren mit einer langsameren Taktfrequenz, das Programm läuft im Prinzip immer noch gleich schnell. D. h. wir brauchen mehr Strom, weil das Programm jetzt länger läuft, ja, weil wenn man es zu langsam runterdreht, dann braucht man wieder mehr Strom, weil die Zeit sich einfach verlängert. Das nennt man Energie to solution. \\
[And: Das hat jetzt weniger mit uns zu tun, ]...\\
Das LRZ hat jetzt auch die Strombeschaffung umgestellt. Die Strombeschaffung früher war immer so, dass  man im Vorhinein geplant hat, eine Woche im Vorhinein oder länger, bzw. dass man ein Jahr, zwei Jahre im Vorhinein diese Basisband kauft, und dann je nach Auslastung den Strom eingekauft hat. Das Probelm ist, dass wir praktisch immer höheren Strom gehabt haben, und der Rechner hat sich irgendwo bewegt, und solang er in einem gewissen Band drinnen ist, ist der Preis normal, nciht billig, weil er bei uns einfahc nicht günstig sit, sondern normal. Das Probelm ist, dass er jedesmals, 
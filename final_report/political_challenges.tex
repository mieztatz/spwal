Due to its distributed nature a nano data center infrastructure will face additional political and legal challenges when compared to a traditional monolithic data center infrastructure. In the following sections we will discuss important political and legal challenges. We acknowledge that there will be many more challenges to be overcome in these fields. However, most of these will arise during implementation and are therefore difficult to predict. Hence, we restricted ourselves to challenges of each field which will certainly have to be overcome.

\subsubsection{Political challenges}
Once the decision to transition to nano data centers has been made the question arises of who will be responsible for that infrastructure. Considering that nano data centers would likely be integrated into today's modems, with which each household connects to their internet service provider (ISP), the ISP would be an obvious choice. The ISP already has knowledge of their network and is also already distributing modems. However, this raises other challenges:\\

\textit{How to distribute data between ISPs?}\\
Most internet users only have one ISP they use to connect to the internet. As not every ISP can be expected to store all of the worldwide information that is currently stored in monolithic data centers, a solution has to be found to access data which is not currently provided by the ISP a person is connected to. This might include international partnerships which further increase the challenge to introduce political policies for such cooperations.\\

\textit{How to ensure net neutrality?}\\
If ISPs are not only responsible for providing bandwidth but also for providing data, a conflict of interest might arise as to which data to provide with which priority. Especially when considering cooperations between different ISPs as mentioned above, an ISP might want to prioritize data deliverance to its own customers before servicing those of other ISPs. To ensure net neutrality, policies have to be introduced to ensure the same quality of service for each customer across different ISPs.\\

%Money?\\

If these challenges prove too difficult to overcome, ISPs might not be the best choice to manage nano data centers. In this case a third party would have to be introduced for managing the nano data center infrastructure. By having an ISP independent third party the issue of net neutrality would be greatly reduced. However, the challenge on how to distribute data across different ISPs would be unaffected. Additionally the question of how to distribute the management software of the nano data center infrastructure onto ISP dependent hardware would arise. 


\subsubsection{Legal challenges}
Legal regulations will pose challenges on the way to a distributed nano data center infrastructure. Some of these regulations are already in effect, others will yet have to be introduced to deal with the new state of the art. In this section we will introduce one challenge each as an example of what kind of legal challenges can be expected. As before, we acknowledge that there will be many other challenges to be overcome.\\

"Location Matters: Limitations of Global-Scale Datacenters" (LocMat) \cite{locmat} mentions the geographic location (jurisdiction) of data as a key problem for using public cloud computing. Privacy laws govern how data is stored and who may access that data. There are many different laws. Transborder data laws govern where data can be stored. Moving data between different jurisdictions is called transborder data flow and can be difficult or illegal. This depends on content type or country of origin. This also intimidates cloud providers, so that they build redundant data centers in each individual jurisdiction. Some jurisdictions allow transborder data crossing, when the other jurisdiction provides at least equivalent levels of protection. The amount and variety of these laws hamper the extension of cloud computing or can even affect the implementation of nano data centers.\\

\textit{The European Union Data Protection Directive (EUDPD). }\\
LocMat \cite{locmat} shows, that holding personal data from an EU citizen in other countries than EU members is forbidden by the EUDPD, except they can provide appropriate protection. Switzerland, Argentina, Canada and the Isle of Man can provide the required protection. This is a very limited region, but in special cases data can also be stored in the United States. In this case the United States Safe Harbour rules play an important role.\\

\textit{The United States Safe Harbor Rules.}\\
With this law and an associated certification companies in the US can process or store EU data. If a company wants to be certified, it has to apply and agree to the requirements. But there are also limitations like the fact that the company has to be under the jurisdiction of the Federal Trade Commission. The in-consequent legislative facts and the complicated laws hamper the idea of global monolithic data centers and also the idea of distributed nano data centers. \cite{locmat}\\

\textit{Conflicting Legislation: The USA PATRIOT ACT.}\\
Obtaining personal data of foreign companies and persons is allowed by this Act, when the data is stored in the US. Therefore, EU companies, which store data in the US, should reckon that the US may read their data. If an US company stores data in the EU, then the Act gives the US also the right to collect private data. This leads to mistrust, insecurity and does not support the idea of a global-scale data center but rather to distributed data centers. \cite{locmat}\\

As pointed out in the previous sections, the current legal situation is complicated and also leaves gaps for uncertainty or mistrust. On the one hand, the laws do not support the introduction of global monolithic data centers, but on the other hand they also do not support the use of nano data centers and a cross-border distribution of data. Global distributed nano data centers could provide greater reach and more opportunities for efficient data storage, distribution and delivery, but this would require a legally secure solution for all involved parties. But even in the future there will be laws that could hinder the idea and implementation of nano data centers.\\


\textit{General Data Protection Regulation (GDPR) of the EU.}
\begin{displayquote}
Wenn ich [...] personenbezogene Daten auf [...] verteilten Systemen speichern will, dann könnte ab 25. Mai nächsten Jahres einer, der jetzt in den Daten gespeichert ist, verlangen: "Wo sind denn meine Daten überall?". [...] Also da sehe ich schon noch Sachen wo ich sage, mit so verteilten Systemen und Rechenzentren wird das relativ schwierig, [...] was nicht heißt, dass das nicht funktioniert, was aber vielleicht heißt, dass der Aufwand das zu realisieren mehr ist als [...] wenn man es bei uns macht \cite{kranzlm}.
\end{displayquote}

% Real quote:
% Wenn ich jetzt da personenbezogene Daten auf so verteilten Systemen speichern will, dann könnte ab 25. Mai nächsten Jahres einer, der jetzt in den Daten gespeichert ist, verlangen, "wo sind denn meine Daten überall?", ja? Also da seh ich schon noch Sachen wo ich sag, mit so verteilten Systemen und Rechenzentren wird des relativ schwierig, ja, was nicht heißt, dass das nicht funktioniert, was aber vielleicht heißt, dass der Aufwand das zu realisieren mehr ist als wie wenn mans bei uns macht \cite{kranzlm}.

On May 25th, 2018 the GDPR EU regulation will come into affect. It states, that any individual has the right to request information about data, which can personally identify them. This includes information about where this data is stored, whether it will be transferred to other entities, how long it will be stored, for which purpose it is stored and others. It also provides an individual with the right to request access, rectification and erasure of this data as well as restricting further processing of that data \cite{gdpr3}. Having a monolithic data center in one physical location makes these requirements easier to fulfil. With a distributed infrastructure however, keeping track of where, what kind of data (personal or not) is stored, (including all its backups) is challenging. Additionally, when considering non EU cooperations of nano data center infrastructures, the organisation handling the personal identifiable data has to ensure that the same regulations as mentioned above are in effect in the country the data will then be stored in \cite{gdpr5}.\\

\textit{Liability for data stored on a nano data center?}\\
Considering that every household that is part of a nano data center infrastructure would physically store some part of the data of the whole system, the question of liability for that part of the data arises. If illegal data is stored on ones own nano data center, can one be held accountable? This problem relates to the "Störerhaftung" (Breach of Duty of Care) regulation in Germany, which, until recently, made operators of open wireless networks accountable for illegal activities performed by users of that network \cite{bgb}. As there is no standardised European or worldwide regulation yet, this will pose a challenge for operators of future nano data center infrastructures.

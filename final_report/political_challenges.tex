Due to its distributed nature a nano data center infrastructure will face additional political and legal challenges when compared to a traditional monolithic data center infrastructure. In the following sections we will discuss important political and legal challenges. We acknowledge that there will be many more challenges to be overcome in these fields. However, most of these will arise during implementation and are therefore difficult to predict. Hence, we restricted ourselves to challenges of each field which will certainly have to be overcome.

\subsubsection{Political challenges}
Once the decision to transition to nano data centers has been made the question arises of who will be responsible for that infrastructure. Considering that nano data centers would likely be integrated into today's modems, with which each household connects to their internet service provider (ISP), the ISP would be an obvious choice. The ISP already has knowledge of their network and is also already distributing modems. However, this raises other challenges:\\

How to distribute data between ISPs?\\
Most internet users only have one ISP they use to connect to the internet. As not every ISP can be expected to store all of the worldwide information that is currently stored in monolithic data centers, a solution has to be found to access data which is not currently provided by the ISP a person is connected to. This might include international partnerships which further increase the challenge to introduce political policies for such cooperations.\\

How to ensure net neutrality?\\
If ISPs are not only responsible for providing bandwidth but also for providing data, a conflict of interest might arise as to which data to provide with which priority. Especially when considering cooperations between different ISPs as mentioned above, an ISP might want to prioritize data deliverance to its own customers before servicing those of other ISPs. To ensure net neutrality, policies have to be introduced to ensure the same quality of service for each customer across different ISPs.\\

%Money?\\

If these challenges prove too difficult to overcome, ISPs might not be the best choice to manage nano data centers. In this case a third party would have to be introduced for managing the nano data center infrastructure. By having an ISP independent third party the issue of net neutrality would be greatly reduced. However, the challenge on how to distribute data across different ISPs would be unaffected. Additionally the question of how to distribute the management software of the nano data center infrastructure onto ISP dependent hardware would arise. 


\subsubsection{Legal challenges}
Legal regulations will pose challenges on the way to a distributed nano data center infrastructure. Some of these regulations are already in effect, others will yet have to be introduced to deal with the new state of the art. In this section we will introduce one challenge each as an example of what kind of legal challenges can be expected. As before, we acknowledge that there will be many other challenges to be overcome.\\

General Data Protection Regulation (GDPR) of the EU.\\
On May 25th, 2018 the GDPR EU regulation will come into affect which states, that any individual has the right to request information about data, which can personally identify them. This includes information about where this data is stored, whether it will be transferred to other entities, how long it will be stored, for which purpose it is stored and many others. It also provides an individual with the right to request access, rectification and erasure of this data as well as restricting further processing of that data \cite{gdpr}.\\

[ab hier erst mal nur Notitzen]
"Wenn ich jetzt da personenbezogene Daten auf so verteilten Systemen [...] speichern will, dann könnte ab 25. Mai nächsten Jahres einer der jetzt in den Daten gespeichert ist verlangen, "wo sind denn meine Daten überall?", ja? Also da seh ich schon noch Sachen wo ich sag, mit so verteilten Systemen und Rechenzentren wird des relativ schwierig, ja, was nicht heißt, dass das nicht funktioniert, was aber vielleicht heißt, dass der Aufwand das zu realisieren mehr ist als wie wenn mans bei uns macht \cite{kranzlm}."

- Who is liable for illegal data stored on a persons nano data center?

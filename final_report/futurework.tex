%inclde idea: use onliney surveys to invite experts to answer questions
In section \ref{sec:challenges} it can be seen that the implementation of a nano data platform not as easy as assumed in \cite{Laoutaris:2008:EEC:1341431.1341442} or other works. There are a few more steps that have to be taken before this can be put into practice. Some issues still have to be investigated and researched in more detail. In our work we figured out the main challenges for the next steps towards a common use of nano date servers.
%future work
For next steps more studies are required regarding technical and non-technical challenges. This work can not be seen as exhaustive, therefore, there might be more challenges than we have considered, especially such of political and legal origin. This is also closely related to the geographic location and the local jurisdiction.

In \ref{sec:model} we have defined the technical challenges. These also partly require further investigations.
On the one hand, this means further investigations into a well-chosen activity time of the nano server in order to achieve better energy efficiency.
Further detailed research is on the other hand also needed for the selection of access networks to investigate which network is most suitable.
Finally, another investigation is needed to find a good balance between the transmission distance and the number of data replicas.


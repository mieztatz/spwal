\documentclass[sigchi-a, authorversion]{acmart}
\usepackage{booktabs} % For formal tables
\usepackage{ccicons}  % For Creative Commons citation icons

\usepackage[toc,page]{appendix}

% Copyright
%\setcopyright{none}
\setcopyright{acmcopyright}
%\setcopyright{acmlicensed}
%\setcopyright{rightsretained}
%\setcopyright{usgov}
%\setcopyright{usgovmixed}
%\setcopyright{cagov}
%\setcopyright{cagovmixed}


% DOI
%\acmDOI{10.475/123_4}

% ISBN
%\acmISBN{123-4567-24-567/08/06}

%Conference
\acmConference[SPWAL LMU]{Wissenschaftliches Arbeiten und Lehren, LMU}{November 2017}{Munich, Germany} 
%\acmYear{1997}
%\copyrightyear{2016}

%\acmPrice{15.00}

%\acmBadgeL[http://ctuning.org/ae/ppopp2016.html]{ae-logo}
%\acmBadgeR[http://ctuning.org/ae/ppopp2016.html]{ae-logo}

\begin{document}
\title{CTNDCI: Identifying the Challenges Towards a distributed Nano Data Center Infrastructure}

\author{Melanie Hauser}
\affiliation{ %
  \institution{Ludwig Maximilian University of Munich}
  \city{Munich}
  \country{Germany}}
\email{melanie.hauser@campus.lmu.de}

\author{Diana Irmscher} 
\affiliation{%
 \institution{Ludwig Maximilian University of Munich}
 \country{Germany}}
\email{d.irmscher@campus.lmu.de}

\author{Mengchu Li}
\affiliation{%
  \institution{Ludwig Maximilian University of Munich}
  \city{Munich} 
  \country{Germany}}
\email{mengchu.li@yahoo.com}

\author{Katrin Kolb}
\affiliation{%
  \institution{Ludwig Maximilian University of Munich}
  \city{Munich} 
  \country{Germany}}
\email{Katrin.Kolb@campus.lmu.de}

\author{Katharina Rupp}
\affiliation{%
  \institution{Ludwig Maximilian University of Munich}
  \city{Munich} 
  \country{Germany}}
\email{katharina.rupp@web.de}

\author{Andreas Scholz}
\affiliation{%
  \institution{Ludwig Maximilian University of Munich}
  \city{Germany}
  \country{Germany}}
\email{andreas.scholz@campus.lmu.de}



\author{Document Identifier: GI 201711}
% The default list of authors is too long for headers.
\renewcommand{\shortauthors}{Katrin Kolb et al.}

\begin{abstract}  % Andreas
In this paper we identify the challenges currently preventing nano data centers from becoming the dominant form of content provision on the internet. With the global increase in IP traffic the question of how to provide and deliver data is becoming increasingly important. Monolithic data centers, as they are used today, pose several problems, such as high energy consumption and lack of scalability. An alternative solution mitigating the problems of monolithic data centers has been proposed in the form of a distributed nano data center infrastructure. Research has shown this to be a superior solution. However, no widespread solution based on a nano data center infrastructure has been implemented as of yet. By identifying the main challenges nano data centers are facing steps can be taken to overcome these challenges in a more focused way, leading to a more economic data distribution.\\
\textit{Author: Andreas Scholz}\\
\end{abstract}

\keywords{Green IT; Nano data center; Energy consumption; Security; Availability; Scalability; Data distribution}

\maketitle

\begin{sidebar}
  \textbf{SPWAL Research Project:} Green Computing
  \textbf{Title:}Identifying the Challenges Towards a distributed Nano Data Center Infrastructure

  \textbf{Progress Report I}

%  dimensions and do not flow the margin text to the
%  next page. 
%  
%  \textbf{Materials:} The margin box must not intrude
%  or overflow into the header or the footer, or the gutter space
%  between the margin paragraph and the main left column. 
%  
%  \textbf{Images \& Figures:} Practically anything
%  can be put in the margin if it fits. Use the
%  \texttt{{\textbackslash}marginparwidth} constant to set the
%  width of the figure, table, minipage, or whatever you are trying
%  to fit in this skinny space.
%
%  \caption{This is the optional caption}
%  \label{bar:sidebar}
\end{sidebar}

%\begin{figure}
%  \includegraphics[width=\marginparwidth]{sigchi-logo}
%  \caption{Insert a caption below each figure.}
%  \label{fig:sample}
%\end{figure}

% Das steht schon ungefähr so in Next Steps, deswegen hab ichs jetzt erst mal rausgelassen aus den Achievements:
%  For more detailed investigations we began grouping the issues by categories. The goals of this are on the one hand being able to name broader fields that still need enhancement, on the other hand omitting some of the categories, e. g. politically motivated or legal obstacles. That way we can focus on problems that are closer related to our fields of study and give a less superficial view in our final report.

\section{Achievements}
We have continued our research into nano data centers according to our project plan. So far our research enabled us to derive some problematic aspects that could prevent a widespread usage of nano data centers. While some of those were explicitly stated, others can be implicitly found through the points the papers lack. The latter bring the risk of making false assumptions, so some statements are yet to be proven by further research. \\
Based on some of the suggestions after our initial project presentation we also decided to broaden our research scope to include not only nano data centers themselves, but also related technologies. In doing so we intend to balance the fact that there is very little research available for nano data centers specifically. By identifying potential problems in more thoroughly researched technologies which are in turn intended to be used in nano data centers we are able to determine whether these problems would still persist in the context of nano data centers.
Aside from broadening our research scope we also drafted a questionnaire aimed at providers of current monolithic data centers (see Appendices on page~\pageref{appendix:quest}). The questionnaire is aimed at identifying areas which are working well in monolithic data centers, but which might pose a problem in the context of nano data centers.\\
\textit{Authors: Team effort}\\

\section{Next Steps}
From a research perspective our next tasks involves narrowing down the scope of the challenges we want to take a closer look at. At the moment our research question is too widely scoped and we will not be able to identify all challenges from all areas relevant to nano data centers (political, environmental, technical, economical, etc.). Our intention is to pick one or two areas and to identify some key challenges within these areas. This has then also to be reflected in our research question and the topic of our paper. It will be challenging to find the right balance between the amount of challenges and the detail in which to analyse them.\\
Regarding the questionnaire we hope to interview someone from the Leibniz Supercomputing Centre(LRZ) in Munich to get some qualitative insights from a professional in the field. Ideally we would like to find some other experts, too. However we expect it will be difficult to achieve this within the tight time-limit, especially considering the holidays coming up.\\
One other idea we had was to develop a second questionnaire to interview ISPs with. As ISPs are the proposed providers and maintainers of a future nano data center architecture, their input would be valuable to explore whether nano data centers are already a topic they are concerned with. However, as with the original questionnaire we are unsure whether we will be able to make according appointments in time.\\
\textit{Authors: Team effort}\\

\section{Deviation from plan}
As our original research plan was based solely on paper research, we deviated from it in so far as we are now also trying to gather qualitative information through interviews. Besides that we did not deviate from our research plan, yet. Considering our proposed next steps however we do acknowledge that further deviations might follow. We consider these deviations necessary to help narrowing down our research question and providing a more focused paper.\\
\textit{Authors: Team effort}\\

% References
\nocite{*}
\bibliography{sigchi-a}
\bibliographystyle{ACM-Reference-Format}

\begin{appendices}
\chapter{Questionnaire}
\begin{enumerate}
\label{appendix:quest}
\item On the website of the LRZ it can be read that \textit{Green IT} is important \cite{LRZGreenIT}. What has been achieved or improved so far?
\item In 2012, the LRZ was awarded the German Data Center Award for \textit{energy and resource efficient data centers} \cite{LRZGreenIT}. What makes the LRZ better on \textit{Green IT} than other data centers?
\item What does the LRZ offer its customers? Are there any special \textit{Green IT} services available? Does the customer have an influence on more environmentally conscious use?
\item Today's use of Internet services has changed massively \cite{TheZetta68:online}. How has the LRZ adapted accordingly?
\item Why are the big data centers still so popular? What are the reasons? Are these political, economic or technical?
\item Have you heard of an alternative solution to monolithic data centers? There are, among others, some research on nano data centers. Does the LRZ also work with these approaches? What is your opinion?
\item How does the LRZ see the data centers of the future? What could be possible? Is it realistic that monolithic data centers could be replaced by special peer-to-peer networks?
\end{enumerate}
\end{appendices}

\end{document}
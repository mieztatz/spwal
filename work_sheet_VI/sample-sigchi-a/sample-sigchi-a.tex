\documentclass[sigchi-a, authorversion]{acmart}
\usepackage{booktabs} % For formal tables
\usepackage{ccicons}  % For Creative Commons citation icons

% Copyright
%\setcopyright{none}
\setcopyright{acmcopyright}
%\setcopyright{acmlicensed}
%\setcopyright{rightsretained}
%\setcopyright{usgov}
%\setcopyright{usgovmixed}
%\setcopyright{cagov}
%\setcopyright{cagovmixed}


% DOI
%\acmDOI{10.475/123_4}

% ISBN
%\acmISBN{123-4567-24-567/08/06}

%Conference
\acmConference[SPWAL LMU]{Wissenschaftliches Arbeiten und Lehren, LMU}{November 2017}{Munich, Germany} 
%\acmYear{1997}
%\copyrightyear{2016}

%\acmPrice{15.00}

%\acmBadgeL[http://ctuning.org/ae/ppopp2016.html]{ae-logo}
%\acmBadgeR[http://ctuning.org/ae/ppopp2016.html]{ae-logo}

\begin{document}
\title{CTNDCI: Identifying the Challenges Towards a distributed Nano Data Center Infrastructure}

\author{Melanie Hauser}
\affiliation{ %
  \institution{Ludwig Maximilian University of Munich}
  \city{Munich}
  \country{Germany}}
\email{melanie.hauser@campus.lmu.de}

\author{Diana Irmscher} 
\affiliation{%
 \institution{Ludwig Maximilian University of Munich}
 \country{Germany}}
\email{d.irmscher@campus.lmu.de}

\author{Mengchu Li}
\affiliation{%
  \institution{Ludwig Maximilian University of Munich}
  \city{Munich} 
  \country{Germany}}
\email{mengchu.li@yahoo.com}

\author{Katrin Kolb}
\affiliation{%
  \institution{Ludwig Maximilian University of Munich}
  \city{Munich} 
  \country{Germany}}
\email{Katrin.Kolb@campus.lmu.de}

\author{Katharina Rupp}
\affiliation{%
  \institution{Ludwig Maximilian University of Munich}
  \city{Munich} 
  \country{Germany}}
\email{katharina.rupp@web.de}

\author{Andreas Scholz}
\affiliation{%
  \institution{Ludwig Maximilian University of Munich}
  \city{Germany}
  \country{Germany}}
\email{andreas.scholz@campus.lmu.de}



\author{Document Identifier: GI 201711}
% The default list of authors is too long for headers.
\renewcommand{\shortauthors}{Katrin Kolb et al.}

\begin{abstract}  % Andreas
In this paper we identify the challenges currently preventing nano data centers from becoming the dominant form of content provision on the internet. With the global increase in IP traffic the question of how to provide and deliver data is becoming increasingly important. Monolithic data centers, as they are used today, pose several problems, such as high energy consumption and lack of scalability. An alternative solution mitigating the problems of monolithic data centers has been proposed in the form of a distributed nano data center infrastructure. Research has shown this to be a superior solution. However, no widespread solution based on a nano data center infrastructure has been implemented as of yet. By identifying the main challenges nano data centers are facing steps can be taken to overcome these challenges in a more focused way, leading to a more economic data distribution.\\
\textit{Author: Andreas Scholz}\\
\end{abstract}

\keywords{Green IT; Nano data center; Energy consumption; Security; Availability; Scalability; Data distribution}

\maketitle

\begin{sidebar}
  \textbf{SPWAL Research Project:} Green Computing
  \textbf{Title:}Identifying the Challenges Towards a distributed Nano Data Center Infrastructure

  \textbf{Progress Report I}
%  dimensions and do not flow the margin text to the
%  next page. 
%  
%  \textbf{Materials:} The margin box must not intrude
%  or overflow into the header or the footer, or the gutter space
%  between the margin paragraph and the main left column. 
%  
%  \textbf{Images \& Figures:} Practically anything
%  can be put in the margin if it fits. Use the
%  \texttt{{\textbackslash}marginparwidth} constant to set the
%  width of the figure, table, minipage, or whatever you are trying
%  to fit in this skinny space.
%
%  \caption{This is the optional caption}
%  \label{bar:sidebar}
\end{sidebar}

%\begin{figure}
%  \includegraphics[width=\marginparwidth]{sigchi-logo}
%  \caption{Insert a caption below each figure.}
%  \label{fig:sample}
%\end{figure}

\section{Achievements}

\section{Next teps}

\section{Deviation from plan}


% References
\bibliography{sigchi-a}
\bibliographystyle{ACM-Reference-Format}

\end{document}
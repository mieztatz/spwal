\section{Introduction} % Katrin


\section{Related Work} % Melanie, Katharina
Some concepts in the field of nano data centers have been developed. For example, Valancious et al. introduced NaDa. NaDa is a distributed computing platform, which uses a managed peer-to-peer model for its infrastructure. They furthermore evaluated their system in terms of energy savings. \\
Jalali et al. \\
Laoutaris et al. \\

\section{Justification} % Melanie, Katharina
Although a lot of research has been done and concepts have been developed, nano data centers have not yet actually been implemented in big extent. To realize nano data centers, it is important to know why it has not been done yet and which obstacles have to be resolved. That is why this paper analyses the causes and obstacles.


\section{Evaluation} % Mengchu


\section{Research Plan} % Diana
\begin{table}[H]
  \caption{Research Plan}
  \label{tab:researchPlan}
  \begin{tabular}{ll}
    \toprule
    17.10.17 & Choosing a SPWL Research Area, join the team \\
    24.10.17 & \\
    06.11.17 & Upload final research proposal \\
    07.11.17 & Presentation of the research proposal \\
    20.11.17 & Upload progress report I \\
    21.11.17 & Progress report I \\
    12.12.17 & Mid term synchronisation \\
    12.01.18 & Upload Progress Report II \\
    26.01.18 & Upload final deliverables \\
    30.01.18 & Presentation of final deliverables \\
    06.02.18 & Presentation of final deliverables \\
    \bottomrule
  \end{tabular}
\end{table}

\section{Risk Analysis} % Diana
Das ist ein Text \cite{DBLP:conf/conext/ValanciusLMDR09} und das auch \cite{DBLP:journals/sigmetrics/JalaliAVHAT14}

\appendix
%Appendix A


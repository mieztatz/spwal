\section{Introduction} % Katrin


\section{Related Work} % Melanie, Katharina
Some concepts in the field of nano data centers have been developed. For example, Valancious et al. introduced NaDa. NaDa is a distributed computing platform, which uses a managed peer-to-peer model for its infrastructure. They furthermore evaluated their system in terms of energy savings. \\
Jalali et al. \\
Laoutaris et al. \\

\section{Justification} % Melanie, Katharina
Although a lot of research has been done and concepts have been developed, nano data centers have not yet actually been implemented in big extent. To realize nano data centers, it is important to know why it has not been done yet and which obstacles have to be resolved. That is why this paper analyses the causes and obstacles.


\section{Evaluation} % Mengchu

\paragraph{Expected achievements of the project:} A list of challenges towards the development of nano data center.
We will first analyse the features that are related to nano data center development and then list the challenges that need to be overcome. 
For a feature to be listed as a challenge, 
the following conditions must be satisfied:
\begin{enumerate}
\item[1.] The feature is a necessary prerequisite for the development of nano data center;
\item[2.] the current status of the feature do not meet the demand of nano data center development. 
\end{enumerate}

\paragraph{Evaluation method:} 
\begin{enumerate}
\item[1.] To test the first condition, 
we will study the existing nano data center models proposed by other research, 
and find out how our proposed challenges are involved in these models. 
For example, whether a proposed challenge is related to the components that construct the infrastructure of these nano data centers,
or which functionality supported by these nano data centers will be influenced by the challenge?
\item[2.] To test the second condition,
we will formulate a report of the current status of the proposed challenges,
and compare the results with their expected status derived from the data center models proposed by other research.
If the current status does not match the expected status,
we will try to find out the reason and propose some approaches to narrow the gap.
\end{enumerate}

\paragraph{Software and resources for the evaluation:}
According to the current evaluation plan, no software needs to be built. Paper survey will be essential for carrying out the evaluation.

\section{Research Plan} % Diana
\begin{table}[H]
  \caption{Research Plan}
  \label{tab:researchPlan}
  \begin{tabular}{ll}
    \toprule
    17.10.17 & Choosing a SPWL Research Area, join the team \\
    24.10.17 & \\
    06.11.17 & Upload final research proposal \\
    07.11.17 & Presentation of the research proposal \\
    20.11.17 & Upload progress report I \\
    21.11.17 & Progress report I \\
    12.12.17 & Mid term synchronisation \\
    12.01.18 & Upload Progress Report II \\
    26.01.18 & Upload final deliverables \\
    30.01.18 & Presentation of final deliverables \\
    06.02.18 & Presentation of final deliverables \\
    \bottomrule
  \end{tabular}
\end{table}

\section{Risk Analysis} % Diana
Das ist ein Text \cite{DBLP:conf/conext/ValanciusLMDR09} und das auch \cite{DBLP:journals/sigmetrics/JalaliAVHAT14}

\appendix
%Appendix A


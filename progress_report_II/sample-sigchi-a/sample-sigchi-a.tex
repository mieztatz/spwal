\documentclass[sigchi-a, authorversion]{acmart}
\usepackage{booktabs} % For formal tables
\usepackage{ccicons}  % For Creative Commons citation icons

\usepackage[toc,page]{appendix}

\setcopyright{acmcopyright}

\usepackage{nameref}


% DOI
%\acmDOI{10.475/123_4}

% ISBN
%\acmISBN{123-4567-24-567/08/06}

%Conference
\acmConference[SPWAL LMU]{Wissenschaftliches Arbeiten und Lehren, LMU}{Januar 2018}{Munich, Germany} 

\begin{document}
\title{CTNDCI: Identifying the Challenges Towards a distributed Nano Data Center Infrastructure \newline {\small Document Identifier: GI 201711}}

\author{Melanie Hauser}
\affiliation{ %
  \institution{Ludwig Maximilian University of Munich}
  \city{Munich}
  \country{Germany}}
\email{melanie.hauser@campus.lmu.de}

\author{Diana Irmscher} 
\affiliation{%
 \institution{Ludwig Maximilian University of Munich}
 \country{Germany}}
\email{d.irmscher@campus.lmu.de}

\author{Mengchu Li}
\affiliation{%
  \institution{Ludwig Maximilian University of Munich}
  \city{Munich} 
  \country{Germany}}
\email{mengchu.li@yahoo.com}

\author{Katrin Kolb}
\affiliation{%
  \institution{Ludwig Maximilian University of Munich}
  \city{Munich} 
  \country{Germany}}
\email{Katrin.Kolb@campus.lmu.de}

\author{Katharina Rupp}
\affiliation{%
  \institution{Ludwig Maximilian University of Munich}
  \city{Munich} 
  \country{Germany}}
\email{katharina.rupp@web.de}

\author{Andreas Scholz}
\affiliation{%
  \institution{Ludwig Maximilian University of Munich}
  \city{Germany}
  \country{Germany}}
\email{andreas.scholz@campus.lmu.de}

% The default list of authors is too long for headers.
\renewcommand{\shortauthors}{Katrin Kolb et al.}

\begin{abstract}
\label{abstract}
%Briefly describe the report's goals, methodologies, and contributions.
In this report, we present our achievements since the last report and describe our plan for concluding the project. We started grouping the issues by categories, conducted the interview with an expert on data centers from the Leibniz Supercomputing Center and elaborated the issues themselves. We proceeded our work as planned. From now on, we finalize the ideas and issues. \\
\textit{Authors: Team effort}\\
\end{abstract}

\keywords{Green IT; Nano data center; Energy consumption; Security; Availability; Scalability; Data distribution}

\maketitle

\begin{sidebar}
  \textbf{SPWAL Research Project:} Green Computing
  \textbf{Title:}Identifying the Challenges Towards a distributed Nano Data Center Infrastructure

  \textbf{Progress Report II}
\end{sidebar}

% Das steht schon ungefähr so in Next Steps, deswegen hab ichs jetzt erst mal rausgelassen aus den Achievements (von Andreas im 19.11.2017 hinzugefügt):
%  For more detailed investigations we began grouping the issues by categories. The goals of this are on the one hand being able to name broader fields that still need enhancement, on the other hand omitting some of the categories, e. g. politically motivated or legal obstacles. That way we can focus on problems that are closer related to our fields of study and give a less superficial view in our final report.

\section{Achievements}
\label{sec:achievements}
% Describe the project's achievement during the reporting period.
We stopped our research into new papers and started analyzing our available information. Hence, we divided the papers up and every team member had to read and summarize one or two papers. We used this method to produce an expert for every paper and also to inform the other team members about the important facts. The resulting summaries are the basis for some chapters in the final paper. This working part was mainly done in independent work by each team member. Since not all papers explicitly provide information about nano data centers but also about related issues, we started grouping the found issues and also began to merge the information. There is very little research available for nano data centers so that combining and drawing conclusions should be well thought out.\\
So far we have constructed an energy consumption model for nano data center based on the energy consumption model proposed in different papers. By analyzing the energy consumption model, we have concluded four technical challenges towards the nano data center development: the activation of nano servers; the selection of the access network that the nano servers are attached to; the location of nano servers; and the data replication strategy.\\
According to our modified project plan from the last report we conducted the interview with an expert on data centers from the Leibniz Supercomputing Center. We designed a questionnaire that was used for this task, but in the course of the interview we had to adjust the questions to the situation and answers of the expert. The interview was recorded, so there is an audio file available. A transcript is in production.\\
We also planned and organized our final paper. This included creating a table of contents and finding important chapters and topics for chapters. We also began to write the chapters including related work and technical challenges.\\
\textit{Authors: Team effort}\\

\section{Next Steps}
\label{sec:next_steps}
% Outline the intended steps to be performed during the next reporting period.
Our next steps will be focused on finalization of our paper. This includes finalizing the description of the two chosen types of challenges relevant to nano data centers (see description in \nameref{sec:achievements}). In this section we will elaborate all challenges that we have found out. In addition, we have to draw up the transcript of the interview. Furthermore, we have to write down our introduction according to the CARS model and to complete our methodology.\\
\textit{Authors: Team effort}\\

\section{Deviation from plan}
\label{deviation_from_plan}
% Should there be any discrepancies between the plan and the reality, describe and justify them. Please indicate any means to adjust the project to the plan, if necessary.
After the deviations presented in the last report, we proceeded our adjusted research plan. Solely, the interview plan changed somewhat, as we adapted the questionnaire to the course of the interview. Some questions didn't fit the course and thus we neglected them. Instead some new questions emerged from the answers and were hence included them. \\
\textit{Authors: Team effort}\\

% References
\nocite{*}
\bibliography{sigchi-a}
\bibliographystyle{ACM-Reference-Format}

\begin{appendices}
\chapter{Questionnaire}
\begin{enumerate}
\label{appendix:quest}
\item On the website of the LRZ it can be read that \textit{Green IT} is important \cite{LRZGreenIT}. What has been achieved or improved so far?
\item In 2012, the LRZ was awarded the German Data Center Award for \textit{energy and resource efficient data centers} \cite{LRZGreenIT}. What makes the LRZ better on \textit{Green IT} than other data centers?
\item What does the LRZ offer its customers? Are there any special \textit{Green IT} services available? Does the customer have an influence on more environmentally conscious use?
\item Today's use of Internet services has changed massively \cite{TheZetta68:online}. How has the LRZ adapted accordingly?
\item Why are the big data centers still so popular? What are the reasons/advantages? Are these political, economic or technical?
\item Are there any disadvantages with monolithic data centers?
\item Have you heard of an alternative solution to monolithic data centers? There are, among others, some research on nano data centers. Does the LRZ also work with these approaches? What is your opinion?
\item In your opinion, what are the advantages and disadvantages of nano data centers?
\item How does the LRZ see the data centers of the future? What could be possible? Is it realistic that monolithic data centers could be replaced by special peer-to-peer networks?
\item Do you think there are any difficulties or special challenges that need to be solved in order to implement nano data centers suitable for the mass or as new state of the art? What are the difficulties oder challenges in your opinion?
\item Do you have any idea or approach how to solve these difficulties or challenges?
\item Would you have an idea for other alternative systems?
\end{enumerate}
\end{appendices}

\end{document}

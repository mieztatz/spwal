\documentclass[sigchi-a, authorversion]{acmart}
\usepackage{booktabs} % For formal tables
\usepackage{ccicons}  % For Creative Commons citation icons

% Copyright
%\setcopyright{none}
\setcopyright{acmcopyright}
%\setcopyright{acmlicensed}
%\setcopyright{rightsretained}
%\setcopyright{usgov}
%\setcopyright{usgovmixed}
%\setcopyright{cagov}
%\setcopyright{cagovmixed}


% DOI
%\acmDOI{10.475/123_4}

% ISBN
%\acmISBN{123-4567-24-567/08/06}

%Conference
\acmConference[SPWAL LMU]{Wissenschaftliches Arbeiten und Lehren, LMU}{November 2017}{Munich, Germany} 
%\acmYear{1997}
%\copyrightyear{2016}

%\acmPrice{15.00}

%\acmBadgeL[http://ctuning.org/ae/ppopp2016.html]{ae-logo}
%\acmBadgeR[http://ctuning.org/ae/ppopp2016.html]{ae-logo}

\begin{document}
\title{Identifying the challenges towards a distributed nano data center infrastructure}

\author{Katrin Kolb}
\affiliation{%
  \institution{Ludwig Maximilian University of Munich}
  \city{Munich} 
  \country{Germany}}
\email{katrinkolb@web.de}

\author{Katharina Rupp}
\affiliation{%
  \institution{Ludwig Maximilian University of Munich}
  \city{Munich} 
  \country{Germany}}
\email{katharina.rupp@web.de}

\author{Mengchu Li}
\affiliation{%
  \institution{Ludwig Maximilian University of Munich}
  \city{Munich} 
  \country{Germany}}
\email{mengchu.li@yahoo.com}

\author{Melanie Hauser}
\affiliation{ %
  \institution{Ludwig Maximilian University of Munich}
  \city{Munich}
  \country{Germany}}
\email{melanie.hauser@campus.lmu.de}

\author{Andreas Scholz}
\affiliation{%
  \institution{Ludwig Maximilian University of Munich}
  \city{Germany}
  \country{Germany}}
\email{andreas.scholz@campus.lmu.de}

\author{Diana Irmscher} 
\affiliation{%
 \institution{Ludwig Maximilian University of Munich}
 \country{Germany}}
\email{d.irmscher@campus.lmu.de}

% The default list of authors is too long for headers.
\renewcommand{\shortauthors}{Katrin Kolb et al.}

  
\begin{abstract} % Andreas
In this paper we identify the challenges currently preventing nano data centers from becoming the dominant form of content provision on the internet. With the global increase in IP traffic the question of how to provide and deliver data is becoming increasingly important. Monolithic data centers, as they are used today, pose several problems, such as high energy consumption and lack of scalability. An alternative solution mitigating the problems of monolithic data centers has been proposed in the form of a distributed nano data center infrastructure. Research has shown this to be a superior solution. However, no widespread solution based on a nano data center infrastructure has been implemented as of yet. By identifying the main challenges nano data centers are facing steps can be taken to overcome these challenges in a more focused way, leading to a more economic data distribution. 
\end{abstract}

\keywords{Green IT; Nano data center; Energy consumption; Security; Availability; Scalability; Data distribution}

\maketitle

%\begin{sidebar}
%  \textbf{Good Utilization of the Side Bar} 
%  
%  \textbf{Preparation:} Do not change the margin
%  dimensions and do not flow the margin text to the
%  next page. 
%  
%  \textbf{Materials:} The margin box must not intrude
%  or overflow into the header or the footer, or the gutter space
%  between the margin paragraph and the main left column. 
%  
%  \textbf{Images \& Figures:} Practically anything
%  can be put in the margin if it fits. Use the
%  \texttt{{\textbackslash}marginparwidth} constant to set the
%  width of the figure, table, minipage, or whatever you are trying
%  to fit in this skinny space.
%
%  \caption{This is the optional caption}
%  \label{bar:sidebar}
%\end{sidebar}

%\begin{figure}
%  \includegraphics[width=\marginparwidth]{sigchi-logo}
%  \caption{Insert a caption below each figure.}
%  \label{fig:sample}
%\end{figure}

\section{Problem Statement} % Andreas
The global IP traffic is continually increasing. An analysis by Cisco shows, that global IP traffic will grow at a Compound Annual Growth Rate (CAGR) of 24 percent from 2016 to 2021 \cite{TheZetta68:online}, of which 71 percent will be delivered by Content Delivery Networks (CDNs) \cite{TheZetta68:online}. To deliver these amounts of data, CDNs make use of large monolithic data centers. These data centers are energy-inefficient, lack scalability and are difficult to deploy. \cite{DBLP:conf/conext/ValanciusLMDR09} A solution mitigating these problems has been proposed by Valancius et al. in 2009 \cite{DBLP:conf/conext/ValanciusLMDR09} in the form of a distributed nano data center infrastructure. However, to date monolithic data centers are still the prevalent solution. It is therefore necessary to identify the reasons why nano data centers are not being actively used yet to enable insights into which challenges have still to be overcome to ensure a transition from monolithic data centers to distributed nano data centers as soon as possible.

\section{Introduction} % Katrin

\section{Related Work} % Melanie, Katharina
Some concepts in the field of nano data centers have been developed. For example, Valancious et al. \cite{DBLP:conf/conext/ValanciusLMDR09} introduced NaDa. NaDa is a distributed computing platform, which uses a managed peer-to-peer model for its infrastructure. They furthermore evaluated their system in terms of energy savings and thus found that NaDa saves minimum 20\% to 30\% in comparison to traditional data centers. \\
In \cite{DBLP:journals/sigmetrics/JalaliAVHAT14} the energy consumption of traditional and nano data centers is compared. Here, nano data centers are described as a very efficient alternative to monolithical data centers.\\
Another example that advocates the usage of nano data centers is ECHOS \cite{Laoutaris:2008:EEC:1341431.1341442}. Although not many words are spent on ECHOS' disadvantages, the paper implies that most of the disadvantages of nano data centers are inherited from its peer-to-peer architecture. Therefore, analyses of those weaknesses have to be taking into consideration, like in \cite{Dumitriu:2005:DoS}, \cite{Mhapasekar:2011:anonymity} and others.
\\

\section{Justification} % Melanie, Katharina
Although a lot of research has been done and concepts have been developed, nano data centers have not yet actually been implemented in big extent. To realize nano data centers, it is important to know why it has not been done yet and which obstacles have to be resolved. That is why the paper will analyse the causes and obstacles.

\section{Evaluation} % Mengchu

\paragraph{Expected achievements of the project:} A list of challenges towards the development of nano data center.
We will first analyse the features that are related to nano data center development and then list the challenges that need to be overcome. 
For a feature to be listed as a challenge, 
the following conditions must be satisfied:
\begin{enumerate}
\item[1.] The feature is a necessary prerequisite for the development of nano data center;
\item[2.] the current status of the feature do not meet the demand of nano data center development. 
\end{enumerate}

\paragraph{Evaluation method:} 
\begin{enumerate}
\item[1.] To test the first condition, 
we will study the existing nano data center models proposed by other research, 
and find out how our proposed challenges are involved in these models. 
For example, whether a proposed challenge is related to the components that construct the infrastructure of these nano data centers,
or which functionality supported by these nano data centers will be influenced by the challenge?
\item[2.] To test the second condition,
we will formulate a report of the current status of the proposed challenges,
and compare the results with their expected status derived from the data center models proposed by other research.
If the current status does not match the expected status,
we will try to find out the reason and propose some approaches to narrow the gap.
\end{enumerate}

\paragraph{Software and resources for the evaluation:}
According to the current evaluation plan, no software needs to be built. Paper survey will be essential for carrying out the evaluation.

\section{Research Plan} % Diana
\begin{table}[H]
  \caption{Research Plan}
  \label{tab:researchPlan}
  \begin{tabular}{ll}
    \toprule
    until 06.11.2017 & Preparation and submission of research proposal \\
    & Literature review to determine SRP's context \\
    & Narrow the scope of the project \\
    \hline
    07.11.2017 & Presentation of the research proposal \\
    & Meeting with the instructor \\
    \hline
    08.11.2017 & Starting project work: \\
    & Research on current nano data centers in practical use \\
    & What has been treated theoretically and practically as nano data centers \\
    & (projects, etc.) \\
    \hline
    until 20.11.2017 & Preparation and submission of progress report I \\
    \hline
    21.11.2017 & Presentation of the progress report I \\
    & Meeting with the instructor \\
    \hline
    22.11.2017 & Continue the project work: \\
    & Research on current nano data centers in practical use \\
    & What has been treated theoretically and practically as nano data centers \\
    & (projects, etc.) \\
    \hline
    12.12.2017 & Mid term synchronisation \\
    & Meeting with the instructor \\
    \hline
    13.12.2017 & Continue the project work: \\
    & Search for obstacles, why the examined works have not yet \\
    & been practically implemented or applied\\
    & Evaluation of the search results \\
    \hline
    until 12.01.2018 & Preparation and submission of progress report II \\
    \hline
    13.01.201\label{key}8 & Continue the project work: \\
    & Evaluation of the search results \\
    \hline
    until 26.01.18 & Preparation and submission of final deliverables \\
    \hline
    30.01.18 or & Presentation of final deliverables \\
    06.02.18 &  \\
    \bottomrule
  \end{tabular}
\end{table}
%Methodik
At the beginning of the project the question is asked, why there are no practical implementations despite some research projects and works on nano data centers.
It is researched whether there is a comparable practical use of distributed nano data centers. It will be searched for corresponding research projects and works. Based on these initial research results, a hypothesis is put forward why there is still no practical use of such distributed systems. This is followed by a search for the reasons why it has not yet come to the use of such nano data centers. The search results will be evaluated at the end.

\section{Risk Analysis} % Diana
There is the possibility that no corresponding literature can be found. In addition, there is a risk that the search lasts too long. Maybe the individual questions in the problem statement can not be answered specifically. The topic could be too comprehensive or too complex, therefore you could stay with the research on specific topics.

% References
\bibliography{sigchi-a}
\bibliographystyle{ACM-Reference-Format}

\end{document}

\documentclass[sigchi-a, authorversion]{acmart}
\usepackage{booktabs} % For formal tables
\usepackage{ccicons}  % For Creative Commons citation icons

% Copyright
%\setcopyright{none}
\setcopyright{acmcopyright}
%\setcopyright{acmlicensed}
%\setcopyright{rightsretained}
%\setcopyright{usgov}
%\setcopyright{usgovmixed}
%\setcopyright{cagov}
%\setcopyright{cagovmixed}


% DOI
%\acmDOI{10.475/123_4}

% ISBN
%\acmISBN{123-4567-24-567/08/06}

%Conference
\acmConference[SPWAL LMU]{Wissenschaftliches Arbeiten und Lehren, LMU}{November 2017}{Munich, Germany} 
%\acmYear{1997}
%\copyrightyear{2016}

%\acmPrice{15.00}

%\acmBadgeL[http://ctuning.org/ae/ppopp2016.html]{ae-logo}
%\acmBadgeR[http://ctuning.org/ae/ppopp2016.html]{ae-logo}

\begin{document}
\title{Identifying the challenges towards distributed nano data center infrastructure}

\author{Katrin Kolb}
\affiliation{%
  \institution{Ludwig Maximilian University of Munich}
  \city{Munich} 
  \country{Germany}}
\email{katrinkolb@web.de}

\author{Katharina Rupp}
\affiliation{%
  \institution{Ludwig Maximilian University of Munich}
  \city{Munich} 
  \country{Germany}}
\email{katharina.rupp@web.de}

\author{Mengchu Li}
\affiliation{%
  \institution{Ludwig Maximilian University of Munich}
  \city{Munich} 
  \country{Germany}}
\email{mengchu.li@yahoo.com}

\author{Melanie Hauser}
\affiliation{ %
  \institution{Ludwig Maximilian University of Munich}
  \city{Munich}
  \country{Germany}}
\email{melanie.hauser@campus.lmu.de}

\author{Andreas Scholz}
\affiliation{%
  \institution{Ludwig Maximilian University of Munich}
  \city{Germany}
  \country{Germany}}
\email{andreas.scholz@campus.lmu.de}

\author{Diana Irmscher} 
\affiliation{%
 \institution{Ludwig Maximilian University of Munich}
 \country{Germany}}
\email{d.irmscher@campus.lmu.de}

% The default list of authors is too long for headers.
\renewcommand{\shortauthors}{Katrin Kolb et al.}

%
% The code below should be generated by the tool at
% http://dl.acm.org/ccs.cfm
% Please copy and paste the code instead of the example below. 
%
\begin{CCSXML}
<ccs2012>
 <concept>
 <concept_id>10003033.10003106.10003110</concept_id>
 <concept_desc>Networks~Data center networks</concept_desc>
 <concept_significance>500</concept_significance>
 </concept>
 <concept>
 <concept_id>10003456.10003457.10003490.10003507.10003508</concept_id>
 <concept_desc>Social and professional topics~Centralization / decentralization</concept_desc>
 <concept_significance>500</concept_significance>
 </concept>
 <concept>
 <concept_id>10011007.10010940.10010971.10011120</concept_id>
 <concept_desc>Software and its engineering~Distributed systems organizing principles</concept_desc>
 <concept_significance>300</concept_significance>
 </concept>
 <concept>
 <concept_id>10010520.10010521.10010537.10003100</concept_id>
 <concept_desc>Computer systems organization~Cloud computing</concept_desc>
 <concept_significance>100</concept_significance>
 </concept>
 <concept>
 <concept_id>10010520.10010521.10010537.10010540</concept_id>
 <concept_desc>Computer systems organization~Peer-to-peer architectures</concept_desc>
 <concept_significance>100</concept_significance>
 </concept>
% <concept>
%  <concept_id>10010520.10010575.10010755</concept_id>
%  <concept_desc>Computer systems organization~Redundancy</concept_desc>
%  <concept_significance>300</concept_significance>
% </concept>
% <concept>
%  <concept_id>10010520.10010553.10010554</concept_id>
%  <concept_desc>Computer systems organization~Robotics</concept_desc>
%  <concept_significance>100</concept_significance>
% </concept>
% <concept>
%  <concept_id>10003033.10003083.10003095</concept_id>
%  <concept_desc>Networks~Network reliability</concept_desc>
%  <concept_significance>100</concept_significance>
% </concept>
</ccs2012>
\end{CCSXML}
\ccsdesc[500]{Networks~Data center networks}
\ccsdesc[500]{Social and professional topics~Centralization / decentralization}
\ccsdesc[300]{Software and its engineering~Distributed systems organizing principles}
\ccsdesc[100]{Computer systems organization~Cloud computing}
\ccsdesc[100]{Computer systems organization~Peer-to-peer architectures}

%\ccsdesc[300]{Computer systems organization~Redundancy}
%\ccsdesc{Computer systems organization~Robotics}
%\ccsdesc[100]{Networks~Network reliability}

\today.
  
\begin{abstract} % Andreas
in IP traffic the question of how to provide and deliver data is becoming increasingly important. Monolithic data centers, as they are used today, pose several problems, such as high energy consumption and lack of scalability. An alternative solution mitigating the problems of monolithic data centers has been proposed in the form of a distributed nano data center infrastructure. Research has shown this to be a superior solution. However, no widespread solution based on a nano data center infrastructure has been implemented as of yet. By identifying the main challenges nano data centers are facing steps can be taken to overcome these challenges in a more focused way, leading to a more economic data distribution. 
\end{abstract}


\keywords{Nano data center; energy consumption; security; availability; scalability; data distribution}

\maketitle

%\begin{sidebar}
%  \textbf{Good Utilization of the Side Bar} 
%  
%  \textbf{Preparation:} Do not change the margin
%  dimensions and do not flow the margin text to the
%  next page. 
%  
%  \textbf{Materials:} The margin box must not intrude
%  or overflow into the header or the footer, or the gutter space
%  between the margin paragraph and the main left column. 
%  
%  \textbf{Images \& Figures:} Practically anything
%  can be put in the margin if it fits. Use the
%  \texttt{{\textbackslash}marginparwidth} constant to set the
%  width of the figure, table, minipage, or whatever you are trying
%  to fit in this skinny space.
%
%  \caption{This is the optional caption}
%  \label{bar:sidebar}
%\end{sidebar}

%\begin{figure}
%  \includegraphics[width=\marginparwidth]{sigchi-logo}
%  \caption{Insert a caption below each figure.}
%  \label{fig:sample}
%\end{figure}

\section{Problem Statement}
The global IP traffic is continually increasing. An analysis by Cisco[0] shows, that global IP traffic will grow at a Compound Annual Growth Rate (CAGR) of 24 percent from 2016 to 2021 [1], of which 71 percent will be delivered by Content Delivery Networks (CDNs) [1]. To deliver these amounts of data, CDNs make use of large monolithic data centers. These data centers are energy-inefficient, lack scalability and are difficult to deploy. [2] A solution mitigating these problems has been proposed by Valancius et al. in 2009 [2] in the form of a distributed nano data center infrastructure. However, to date monolithic data centers are still the prevalent solution. It is therefore necessary to identify the reasons why nano data centers are not being actively used yet to enable insights into which challenges have still to be overcome to ensure a transition from monolithic data centers to distributed nano data centers as soon as possible.

\section{Introduction} % Katrin

\section{Related Work} % Melanie, Katharina
Some concepts in the field of nano data centers have been developed. For example, Valancious et al. \cite{DBLP:conf/conext/ValanciusLMDR09} introduced NaDa. NaDa is a distributed computing platform, which uses a managed peer-to-peer model for its infrastructure. They furthermore evaluated their system in terms of energy savings and thus found that NaDa saves minimum 20\% to 30\% in comparison to traditional data centers. \\
In \cite{DBLP:journals/sigmetrics/JalaliAVHAT14} the energy consumption of traditional and nano data centers is compared. Here, nano data centers are described as a very efficient alternative to monolithical data centers.\\
Another example that advocates the usage of nano data centers is ECHOS \cite{Laoutaris:2008:EEC:1341431.1341442}. Although not many words are spent on ECHOS' disadvantages, the paper implies that most of the disadvantages of nano data centers are inherited from its peer-to-peer architecture. Therefore, analyses of those weaknesses have to be taking into consideration, like in \cite{Dumitriu:2005:DoS}, \cite{Mhapasekar:2011:anonymity} and others.
\\

\section{Justification} % Melanie, Katharina
Although a lot of research has been done and concepts have been developed, nano data centers have not yet actually been implemented in big extent. To realize nano data centers, it is important to know why it has not been done yet and which obstacles have to be resolved. That is why the paper will analyse the causes and obstacles.

\section{Evaluation} % Mengchu

\paragraph{Expected achievements of the project:} A list of challenges towards the development of nano data center.
We will first analyse the features that are related to nano data center development and then list the challenges that need to be overcome. 
For a feature to be listed as a challenge, 
the following conditions must be satisfied:
\begin{enumerate}
\item[1.] The feature is a necessary prerequisite for the development of nano data center;
\item[2.] the current status of the feature do not meet the demand of nano data center development. 
\end{enumerate}

\paragraph{Evaluation method:} 
\begin{enumerate}
\item[1.] To test the first condition, 
we will study the existing nano data center models proposed by other research, 
and find out how our proposed challenges are involved in these models. 
For example, whether a proposed challenge is related to the components that construct the infrastructure of these nano data centers,
or which functionality supported by these nano data centers will be influenced by the challenge?
\item[2.] To test the second condition,
we will formulate a report of the current status of the proposed challenges,
and compare the results with their expected status derived from the data center models proposed by other research.
If the current status does not match the expected status,
we will try to find out the reason and propose some approaches to narrow the gap.
\end{enumerate}

\paragraph{Software and resources for the evaluation:}
According to the current evaluation plan, no software needs to be built. Paper survey will be essential for carrying out the evaluation.

\section{Research Plan} % Diana
\begin{table}[H]
  \caption{Research Plan}
  \label{tab:researchPlan}
  \begin{tabular}{ll}
    \toprule
    17.10.17 & Choosing a SPWL Research Area, join the team \\
    24.10.17 & \\
    06.11.17 & Upload final research proposal \\
    07.11.17 & Presentation of the research proposal \\
    20.11.17 & Upload progress report I \\
    21.11.17 & Progress report I \\
    12.12.17 & Mid term synchronisation \\
    12.01.18 & Upload Progress Report II \\
    26.01.18 & Upload final deliverables \\
    30.01.18 & Presentation of final deliverables \\
    06.02.18 & Presentation of final deliverables \\
    \bottomrule
  \end{tabular}
\end{table}

\section{Risk Analysis} % Diana
Das ist ein Text \cite{DBLP:conf/conext/ValanciusLMDR09} und das auch \cite{DBLP:journals/sigmetrics/JalaliAVHAT14}

% References
\bibliography{sigchi-a}
\bibliographystyle{ACM-Reference-Format}

\end{document}
